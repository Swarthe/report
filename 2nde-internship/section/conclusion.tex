\section{Conclusion}

\subsection{Employee Life - Comparison}

As mentionned in \hyperlink{subsection.2.4}{section 2.4}, work life at ATP was
generally pleasant and friendly. I sought to compare this experience to the
french analogue by asking fellow student Arthur Bourdeaut how the work
environment was during his internship. Below is his reply when he was asked
asked.

\begin{quote}
    \color{dgreen} they [the employees] were nice, [...] and the other
    stagiaires\footnote[0]{French for "interns"} (approximately 25 years old)
    helped me

    \color{black} -- Arthur Bourdeaut
\end{quote}

Thus, from my limited information, the work environments in France and Denmark
are comparable in terms of the previously mentionned qualities (atmosphere,
employee relations etc.).

\subsection{General Observations}

I made a multitude of observations (listed below) and gained much knowledge that
would have otherwise been impossible to learn without practical experience.

\begin{itemize}
    \item That the administrative complexity and inflexibility of large
          companies sometimes renders them unable to modernise or modify old and
          complex systems despite the possible benefits.
    \item How practical implementations are very different from the theoretical
          visions, as real world circumstances can completely change the
          circumstances and methods of putting into action the plans of systems
          architects.
    \item The multitude of completely different systems interacting with each
          other that sustain large and diverse companies as mentionned in
          \hyperlink{subsection.2.1}{subsection 2.1}. For example, SQL may be
          used to manage a MariaDB database containing data collected from a
          front end (website) using javascript and running on Windows Server
          2019. This kind of chain can be a part of a larger web of dependencies
          and tools, which systems architects keep track of so as to potentially
          improve or maintain them.
    \item How the simple physical yet highly physical task of managing and
          organising hardware is much more active and complicated than it may
          seem, as computer hardware in itself is a complicated realm.
    \item How to most efficiently move a Lenovo ThinkCentre M715q Ryzen 3 2200GE
          4GB 500GB Windows 10 Home Desktop PC from the storage room to a cart.
\end{itemize}

Thanks to this type of knowledge, I generally have a more accurate perception of
how large companies are structures, of how IT functions and can function in
companies and how employees interact with each other and adapt to problems. Of
course, I am now relatively familiar with the profession of Systems
Architecture, which was nearly completely unknown to me previously and is a
consideration for my career.
